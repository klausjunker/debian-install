\section{Vim}



\subsection{ASCII-Code eingeben}
im Insert-Modus:
\begin{verbatim}
 ctrl+v  ... ⏎ (dez)     
 ctrl+v  x.. ⏎ (hex)    
	
 Beispiele: 
	 ctrl+v 65 <space> ergibt:	A
	 ctrl+v 065  ergibt:	A
	 ctrl+v 223  ergibt:	ß
	 ctrl+v x61  ergibt:	a
	 ctrl+v <ESC> ergibt  ^ [    (ohne Leerzeichen!)
		
\end{verbatim}



\subsection{UTF-8 eingeben}
\begin{verbatim}
 :dig			
 Ctrl+k  + 2Zeichencode 
 
 Beispiel: Ctrl+k NO  ergibt:	¬
\end{verbatim}

\subsection{Unicode eingeben}
 (im Insert-Modus):
\begin{verbatim}
 ctrl+v u .... ⏎  (hex)   
 ctrl+v U .... ⏎  (hex)   

 Beispiele:
	 Return 23CE						 ctrl + u + 23ce ergibt: ⏎
	 Violinschlüssel: 1D11E  ctrl + U + 0001 + d11e  
	 Euro:									 ctrl+v u 20ac		 ergibt:	€

\end{verbatim}
Bemerkung:  Den Violinschlüssel  %𝄞 
kann Latex aber nicht verarbeiten!

\subsection{Hexmode ein/aus:}
\begin{verbatim}
 :%!xxd
 :%!xxd -r
\end{verbatim}

\subsection{vim beenden}

in .vimrc:

\begin{verbatim}
 :autocmd VimLeave * silent !clear 
\end{verbatim}




\subsection{Anzeigen des Zeichen-Codes}

Cursor auf dem Zeichen)

mit:  ga  $\rightarrow$ Hexcode (Ascii/Unicode)

mit g8:  UTF-8 Kodierung


1.Beispiel:

ß   ga $\rightarrow$ 223(dez) 00DF(hex) g8 $\rightarrow$ c3 9f 

binär: 000 1101 1111

2.Block (von 0080-07FF): 110x xxxx 10xx xxxx

110(0 0011) 10(01 1111) $\rightarrow$ c3 9f

2.Beispiel:

⏎  ga $\rightarrow$ 23ce g8 $\rightarrow$ e2 8f 8e

binär: 0010 0011 1100 1110

3. Block von (0800-FFFF): 1110 xxxx 10xx xxxx 10xx xxxx

1110 (0010) 10(00 1111) 10(00 1110) $\rightarrow$ E2 8F 8E

\subsection{Variablen in Vim}

\begin{verbatim}
 let @a="Text"
 echo @a
\end{verbatim}

\begin{verbatim}
:!echo % && echo %:r  

:help  filename-modifiers
\end{verbatim}

\subsection{Register}

\begin{verbatim}
:reg

kopieren in reg:		"ayy
einfügen aus reg:		"ap

\end{verbatim}

\subsection{Buffers}
\begin{verbatim}
:help buffers

:badd f1.txt
\end{verbatim}

