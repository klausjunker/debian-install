\section{WLAN}




\subsection{aptitude}
\begin{enumerate}
\item rfkill
\item wireless-tools: iwconfig iwlist
\item wpasupplicant 
\item firmware-b43-installer (b43-fwcutter)
\end{enumerate}

{\texttt dpkg-reconfigure firmware-b43-installer}

\subsection{Hardware}


\begin{itemize}
\item 
{\texttt lspci -nnk $|$ grep -i wlan}

{\texttt lshw -c network}
\item 
{\texttt modprobe -rf b43}

{\texttt modprobe  b43}

\item 
{\texttt rfkill list}

Ergebnis soll sein: 

Hardblocked: no (im BIOS: abstellen)

Dell D531 Wireless 

Bluetooth no

internal-wifi yes

wireless hotkey none

Lan/wifi auto switch off

Softblocked: no

rfkill unblock 0  (bzw. 1 ... )
\item

{\texttt ip link set dev wlan0 up}

\item
{\texttt iwconfig}

Ergebnis: wlan0 ESSID: ......


{\texttt iw dev wlan0 link}

\end{itemize}

\subsection{/etc/network/interfaces}


\begin{verbbox}
allow-hotplug wlan0
iface wlan0 inet static
    address 192.168.1.x/24
    gateway 192.168.1.1
    wpa-ssid ...
    wpa-psk  ......
\end{verbbox}
\fbox{\theverbbox}

\begin{verbbox}
allow-hotplug wlan0
iface wlan0 inet dhcp 
    wpa-ssid ...
    wpa-psk  ......
\end{verbbox}
\fbox{\theverbbox}

Kabel-Anschluss:

Welche Schnittstelle hat die Netzwerkkarte?
{\texttt cat /proc/net/dev}



\begin{verbbox}
#allow-hotplug eth0
auto eth0
iface eth0 inet static
		address 172.16.1.105/24
		gateway 172.16.1.1
\end{verbbox}
\fbox{\theverbbox}



\subsection{Wpa}

\texttt{wpa\_passphrase SSID PW $>>$ wpa.txt}

\texttt{chmod 600 /etc/network/interfaces}


\subsection{/etc/resolv.conf}
\begin{verbbox}
search asdf
nameserver 192.168.1.1
\end{verbbox}
\fbox{\theverbbox}


\subsection{WLAN-Router-Schule}


WLAN (dlink-dir600): 172.16.1.104

Gateway: 172.16.1.1

Adress-Raum Notebooks Junker: 192.168.1.*
 
SSID: OpenWrt-jk

password: ...

IP intern: 192.168.1.1  

root-password(root): ...

auch mit ssh root@192.168.1.1: 

wpa2personal

to do: MAC-Filter, dhcp, ...
 
