
\documentclass[12pt,pdftex]{article}
\usepackage{jkka}[2022/11/01] 
\usepackage{jklisting}
\usepackage{jkexperiment}
\usepackage{jklistingstyle}
\setcounter{jkdebuglevel}{0}
\setcounter{jkci}{0}
\lfoot{Junker} 	
%------------------------------------------------------------------
\lhead{KA 3} 
\rhead{4.3.2024}
\rfoot{JS1-M1}
\DTLloaddb{liste}{../m5.k3.csv}
\cfoot{}
%------------------------------------------------------------------
\begin{document}
\input{a/b.tex}
\DTLforeach{liste}{\sKlasse=Klasse,\sVN=Vorname,\sNN=Nachname,\sSID=Sid}
{
	\addtocounter{jkci}{1}  % laufende Schülernummer !!!
	\chead{\thejkci\,(\sSID) - \sVN \,  \sNN \, - \sKlasse}
	\setcounter{jkaufgabe}{0}
Hinweis: Die Aufgaben können zufällig, also nicht nach Schwierigkeitsgrad sortiert, sein!


Der Rechenweg muss erkennbar sein! 

\par\bigskip
%	\input{a/a1.tex}
	\input{a/a1-integrale.tex}
\par\bigskip
	\input{a/a2-integrale.tex}
\par\bigskip
	\input{a/a3-integrale.tex}
\par\bigskip
	\input{a/a4-integrale.tex}
\par\bigskip
\newpage
	\input{a/a5-integrale.tex}
\par\bigskip
	\input{a/a6-integrale.tex}
\par\bigskip
	\input{a/a7-integrale.tex}
\par\bigskip
	\input{a/a8-integrale.tex}
\par\bigskip
	\input{a/a9-integrale.tex}
\newpage
	\input{a/a10-integrale.tex}
\par\bigskip
	\input{a/a11-optimierung.tex}

\newpage
{\Huge Wahlaufgaben: }\par
Wählen Sie eine der beiden folgenden Aufgaben!
\par\bigskip
	\input{a/a12-optimierung.tex}
\par\bigskip
	\input{a/a13-rotationsvolumen.tex}

\newpage
%	\vspace{6cm}
%\par\bigskip\hrule\bigskip
%	\input{a/a3.tex}
%	\vspace{6cm}
%\par\bigskip\hrule\bigskip
%    \input{b/a-funktionen001.tex}
%\newpage
\newpage
}	
\end{document}	
